
\section{Task E}
\subsection{Two-cell environment}
\subsubsection{Environment description}
\subsubsection{Agent description}

\subsubsection{Random dirt performance test}
This test case is the same as previous, but here the environment will receive
randomly generated dirt at the start of each step before the agent perceives the
location.  Then if it has covered all the cells, it will stop and wait for
a maximum of 3 steps or until its current location becomes dirty. Then it will
restart it self and go over the environment once more. This continues until the
number of steps has reached the maximum amount.  They stay very similar because
of the RNG-method chosen. Which is using the ctime library, seed the RNG with
time(0), and generate a random number which is limited with modulus 5.

Because of this modulus 5, it will generate dirt extremely often and thus
resulting in a low performance.  A clean cell will because of this have a 20\%
chance of becoming dirty after each step.

Each test is run three times to yield some comparison of its randomized dirt
generation.

\begin{longtable}{p{0.05\textwidth} p{0.075\textwidth} p{0.075\textwidth} 
									p{0.05\textwidth} p{0.075\textwidth} p{0.075\textwidth} 
									p{0.13\textwidth}}
Start	& A & B & Steps & Moves & Cleans & Performance \\\hline
A & Clean & Clean & 1000 
		 & 239 & 279 & 1480 \\
	&&&& 233 & 305 & 1404 \\
	&&&& 241 & 292 & 1423 \\
	&&&& 238 & 287 & 1378 \\
	&&&& 231 & 309 & 1426 \\\hline
  &&& AVG & 236.4 & 294.4 & 1422.2\\\hline

B & Clean & Clean & 1000 
		 & 236 & 279 & 1437 \\
	&&&& 236 & 317 & 1412 \\
	&&&& 236 & 306 & 1412 \\
	&&&& 240 & 290 & 1478 \\
	&&&& 244 & 308 & 1474 \\\hline
  &&& AVG & 238.4 & 300.0 & 1442.6 \\\hline
\end{longtable}

\subsection{Simple environment}
\subsubsection{Environment description}

\subsubsection{Agent description}

\subsubsection{Performance test - updated dirt}
For the performance test we created an environment of 5 rows consisting of 6
cells each.  The outer layer of cells are set to walls and there is no obstacles
in the environment.  The dirt is randomly generated from the start and the agent
does not maintain a log of the environment, nor where it's been. It follows the
algorithm by finding the corner and then continuing by making passes side to
side and work its way up and down.  It will pause each time it has found all
four corners, the pause is a maximum of 3 steps or untill the location becomes
dirty.

The grid is a 5 by 6 matrix, outer layer is walls and the inner area is a 3 by 4
matrix where dirt is randomly generated.

In this case the chances of a clean cell becoming dirty again during a step is
20\%\.  This means the rate of clean cells becoming dirty again is extremely
likely and therefore resulting in such a low performance since the perfect
performance would be 12.000.

\begin{longtable}{ p{0.05\textwidth} p{0.075\textwidth} p{0.075\textwidth} 
									 p{0.13\textwidth} }
Steps & Moves & Cleans	& Performance \\\hline
1000	& 525 & 444 & 2214 \\
 		 	& 535 & 434 & 2102 \\
 			& 537 & 431 & 2197 \\
 			& 531 & 438 & 2163 \\
 			& 522 & 447 & 2328 \\\hline
AVG		& 530 & 438 &	2200 \\\hline
\end{longtable}


\subsection{Recursive agent - Theoretical}
\subsubsection{Environment description}

\subsubsection{Agent description}

\subsubsection{Performance test - updated dirt}


