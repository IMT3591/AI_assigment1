
\section{Task E}
\subsection{Two-cell environment}
\subsubsection{Environment description}
\t This is the basic environment, with only two cells. The two cells are manually
created and they only point one to each other. This means that the only neighbor
of the left cell is the right cell and viceversa.

The dirty state of each cell is randomly updated with a probability of the 20\% 
of getting dirty. One out of 5 times the cell gets dirty.

\subsubsection{Agent description}
\t The agent implementation for the two cell environment is easier than for
bigger or more complex environments. 

In this case the agent has less states. It can be placed in a clean cell working,
then it will move to the other cell. If it is in a dirty cell it will suck it. 
If it is off he will do nothing.

\subsubsection{Random dirt performance test}
This test case is the same as previous, but here the environment will receive
randomly generated dirt at the start of each step before the agent perceives the
location.  Then if it has covered all the cells, it will stop and wait for
a maximum of 3 steps or until its current location becomes dirty. Then it will
restart it self and go over the environment once more. This continues until the
number of steps has reached the maximum amount.  They stay very similar because
of the RNG-method chosen. Which is using the ctime library, seed the RNG with
time(0), and generate a random number which is limited with modulus 5.

Because of this modulus 5, it will generate dirt extremely often and thus
resulting in a low performance.  A clean cell will because of this have a 20\%
chance of becoming dirty after each step.

Each test is run three times to yield some comparison of its randomized dirt
generation.

\begin{longtable}{p{0.05\textwidth} p{0.075\textwidth} p{0.075\textwidth} 
									p{0.05\textwidth} p{0.075\textwidth} p{0.075\textwidth} 
									p{0.13\textwidth}}
Start	& A & B & Steps & Moves & Cleans & Performance \\\hline
A & Clean & Clean & 1000 
		 & 239 & 279 & 1480 \\
	&&&& 233 & 305 & 1404 \\
	&&&& 241 & 292 & 1423 \\
	&&&& 238 & 287 & 1378 \\
	&&&& 231 & 309 & 1426 \\\hline
  &&& AVG & 236.4 & 294.4 & 1422.2\\\hline

B & Clean & Clean & 1000 
		 & 236 & 279 & 1437 \\
	&&&& 236 & 317 & 1412 \\
	&&&& 236 & 306 & 1412 \\
	&&&& 240 & 290 & 1478 \\
	&&&& 244 & 308 & 1474 \\\hline
  &&& AVG & 238.4 & 300.0 & 1442.6 \\\hline
\end{longtable}

\subsection{Simple environment}
\subsubsection{Environment description}\label{sec:SimpleEnvDesc}
\t We created an environment of 5 rows consisting of 6 cells each. The outer layer of
cells are set to walls and there are no obstacles within. That makes a rectangular
inner shape of 3 rows and 4 columns. 

The dirt spread around the environment is randomly generated every time the main
program updates. The probability that a cell gets dirty is 20\%\. This means that the
cell will become dirty once every 5 times it gets updated.


\subsubsection{Agent description}
\t The agent in the simple environment it's the basic agent we programmed called
dummyAgent aka Bender. This agent uses the S algorithm to go trhough the hole
environment.

In the simple environment as explained in section \ref{sec:SimpleEnvDesc} The dirt
 state in the cell is updated. With this condition, the agent stops working once he
 found all the four corners. When the current location becomes dirty or he has being
 waiting for 3 cycles he will reboot and start cleaning again. By waiting this time 
 we make sure that the agent doesn't move over just cleaned cells.

\subsubsection{Performance test - updated dirt}
\t For the performance test we created an environment of 5 rows consisting of 6
cells each.  The outer layer of cells are set to walls and there is no obstacles
in the environment.  The dirt is randomly generated from the start and the agent
does not maintain a log of the environment, nor where it's been. It follows the
algorithm by finding the corner and then continuing by making passes side to
side and work its way up and down.  It will pause each time it has found all
four corners, the pause is a maximum of 3 steps or untill the location becomes
dirty.

The grid is a 5 by 6 matrix, outer layer is walls and the inner area is a 3 by 4
matrix where dirt is randomly generated.

In this case the chances of a clean cell becoming dirty again during a step is
20\%\.  This means the rate of clean cells becoming dirty again is extremely
likely and therefore resulting in such a low performance since the perfect
performance would be 12.000.

\begin{longtable}{ p{0.05\textwidth} p{0.075\textwidth} p{0.075\textwidth} 
									 p{0.13\textwidth} }
Steps & Moves & Cleans	& Performance \\\hline
1000	& 525 & 444 & 2214 \\
 		 	& 535 & 434 & 2102 \\
 			& 537 & 431 & 2197 \\
 			& 531 & 438 & 2163 \\
 			& 522 & 447 & 2328 \\\hline
AVG		& 530 & 438 &	2200 \\\hline
\end{longtable}

\subsection{Recursive Agent}
\subsubsection{Environment description}
In this setting the environments dirt allocation is randomly generated whenever
the agent moves to a new location.  It goes through the list sent as a parameter
to the visit function to allocate dirt.


\subsubsection{Agent description}
The environments shape, dirt allocation will not affect the agent, since it is
doesn't make any assumptions about the environment. The agent will 

\subsubsection{Performance test - updated dirt}
\begin{longtable}{ p{0.05\textwidth} p{0.075\textwidth} p{0.075\textwidth} 
									 p{0.13\textwidth} }
Steps & Moves & Cleans	& Performance \\\hline
1000	&  &  &  \\
 		 	&  &  &  \\
 			&  &  &  \\
 			&  &  &  \\
 			&  &  &  \\\hline
AVG		&  &  &	 \\\hline
\end{longtable}



